\documentclass[twoside,a4paper,11pt]{article}
\usepackage[a4paper,top=2.54cm,bottom=2.54
cm,left=3cm,right=3cm,marginparwidth=1.75cm ]{geometry}
\usepackage{graphicx}
\usepackage{multicol}
\usepackage{ragged2e}
\setlength{\columnsep}{1cm}
\usepackage{fancyhdr}\fancyhf{}
\lhead{}
\rhead{\thepage}
\cfoot{}
\fancypagestyle{plain}{}
\usepackage[english]{babel}
\usepackage[utf8]{inputenc}
\usepackage{lipsum}
\title{Gamification of Electric
 Circuits Education}
\author{Nour Gaser , Hazem Gamal , Rokaia Medhat  }

\date{December 2022}

\begin{document}
\begin{center}
\LARGE
\textbf{Gamification of Electric Circuits Education}
\newline
\hfill \break
\large
Nour Gaser, Rokaia Medhat, Mohamed Hussein \& Hazem Gamal
 \newline
\hfill \break
\normalsize
Misr University for Science and Technology -\textit{ MUST}, College of Computers and Artificial Intelligence Technology - \textit{CAIT}, Department of Computer Science - \textit{CS}

\hfill \break
\normalsize
\large 2022-2023   
\end{center}
\pagestyle{fancy}
\setlength{\headheight}{1.5cm}.

\centering
\hrule
\subsection*{Abstract} 
\footnotesize
\centering
In the current academic and research scene, the topics of embedded systems and robotics are of high interest and need, though learners often find that they lack the needed fundamentals in electronics to conduct their projects and research. This project proposes a visual education solution for fundamental electronics and electric circuit simulation, by harvesting the concepts and technologies of gamification, game design, and puzzle design. 

\hfill \break
\hrule

\begin{multicols}{2}
\footnotesize
\justifying
\section{Introduction}

Most students regardless of their scientific level often find the idea of learning about circuits a very intimidating one,it is a topic often associated with the idea of being difficult. And that applies to both the theoretical side and the practical side of the learning process, it is a demanding topic that requires the student to have an understanding in multiple fields such as mathematics and physics . 

Students from all levels may find difficulties in understanding topics such as voltage, current, and how these properties are affected by different components in a circuit. This can be seen for example in the research administered by Carol Bowman and Gordon J. Aubrecht, II [1], they found that students find the concept of voltage very confusing, and that is due to the fact that students first start by learning about current, then when they start learning about voltage they apply the idea of flow which applies to current, as a result students start confusing the two concepts together, and that it is despite careful texts which attempt to clarify the difference among the two.

Another challenge that students face nowadays is their very short attention spans. Students need something engaging and entertaining to keep them interested, traditional methods such as long lectures, and large textbooks, while informative, but they lack the attention grabbing element, students get bored and lose interest. According to Neil A. Bradbury [17] several institutions have brought down the length of lectures to only 15 minutes. This is based on the belief that a lecture any longer than 15 minutes is not going to be effective for students.
\vfill
\section{Gamification as an approach}
There have been a lot of different definitions for gamification with different perspectives from different authors. Dixon, Khaled, and Nacke suggested defining “gamification” as “the use of game design elements in non-game contexts”. van Grove(2011) [2] ”Gamification is to change something that is not a game through a game or its elements.”. MacMillan (2011) [3] ” Gamification, defined as the use of game mechanics, dynamics, and frameworks to promote desired behaviors”. So we could simply say that gamification is the systematic process of applying game mechanics to non-game contexts to make difficult tasks more enjoyable. 

Gamification aims to make an otherwise dull experience tolerable, if not desirable. Good game design must be practiced, otherwise either the essence of the topic will be lost (preserve the essence of the topic), or the experience will just be dull, or even leaving a negative impression on the topic.

\section{Our proposed solution}

Our proposed solution is to develop an educational puzzle game, whose player can both enjoy an entertaining experience regardless of their scientific interest level, all the while implicitly gaining valuable experience and knowledge in electric circuits design, where they can opt-in for a more academic learning experience through references to external materials, data-sheets, and circuit schematics designed to encourage real-life experimentation, and allow for academic instructors and supervisors to supplement their coursework and practical assignments with select (or all) sections of the game.
\section{Preliminary study}
\subsection{Literature review}
\lipsum[0-1]
\subsection{Case studies}
\lipsum[0-1]
\subsection{Survey}
\lipsum[0-1]
\section{Methodology}
\lipsum[0-1]
\section{Results}
\lipsum[0-1]
\section{Conclusion}
\lipsum[0-1]
\section{Refrences}
\lipsum[0-1]
\end{multicols}


\end{document}
